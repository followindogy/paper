\chapter{基于优化CPU和GPU内存交互的设计和实现}
前两章总结了大数据高通量仿真中CPU/GPU异构并行计算的必要性和一些已有的GPU异构编程优化方法,并利用编译
原理中引用-定值链和定值-引用链分析程序中的数据流,介绍了一些适用于一般程序的代码移动优化方法。
程序员只有清楚的了解CPU/GPU的体系结构特点,再根据具体的应用场景设计合适的程序逻辑,选择合适的GPU编程
优化方法才能够编写出充分利用GPU计算性能的程序。本章基于CPU/GPU异构计算的内存特点,结合编译原理中的代码
移动规则新提出了一种通过修改源代码来减少CPU和GPU内存之间数据交换带来的系统消耗的优化方法,并在第四章
通过一些对比实验证明了本文提出的方法确实有优化GPU程序的作用。